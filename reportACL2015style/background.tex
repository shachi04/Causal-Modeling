%!TEX root = causalityRetweetPropensity.tex
\section{Related Work}
\label{sec:background}
\begin{itemize}
\item homophily paper
\item COSN conf paper
While meta data metrics are typically focused on, there are previous studies that have been conducted purely on examining the content of the tweet. Tsugawa and Ohsaki examined tweets sentiment level in relation to how "viral" a tweet was. Virality of a tweet was measured by the number of meesages that were retweeted and the time elapsed from the original posting. They found that negative tweets, text that was classified as having a negative sentiment, had a more rapid and frequent retweet than positive tweets. Negative messages were found to be retweeted by a factor of 1.2-1.6 times more and would be retweet quicker at a rate of 1.25 faster. This work suggests, rather expectedly, that the content of a tweet will effect the rate and amount of retweets of that message. While acknowledging that content is an important factor in retweeting, this paper will examine sentiment within the context of causal structure discovery, but will only examine possible social effects for causal modeling. 
\item Truthy - papers
\end{itemize}